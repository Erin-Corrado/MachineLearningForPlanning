\documentclass[tog]{acmsiggraph}

% haha wow, we have so many \thanks{}es that we ran out of symbols to
% use. this is set by the documentclass, so this is sorta breaking the
% rules, but hey, the alternative is to break compilation.
\makeatletter
\let\@fnsymbol\@arabic
\makeatother

%%% Used by the ``review'' variation; the online ID will be printed on
%%% every page of the content.
\TOGonlineid{45678}

%%% Used by the ``preprint'' variation.
\TOGvolume{0}
\TOGnumber{0}

\newcommand{\email}[1]{\href{mailto:#1}{\nolinkurl{#1}}}
\newcommand{\emailfoot}[1]{\thanks{\email{#1}}}

\title{S2: STAR Report Introduction, Taxonomy, and Body}

\author{Students: %
 Sanjivi Muttena\emailfoot{sanjivi.muttena@rutgers.edu}, %
 Nikhil Kumar\emailfoot{nikhilkumar516@gmail.com}, %
 Erin Corrado\emailfoot{e.corrado144@gmail.com}, %
 Daniel Bordak\emailfoot{dbordak@fastmail.fm},%
 \\Zooraze Tariq\emailfoot{zooraze@gmail.com}, %
 James Lee\emailfoot{yl50@scarletmail.rutgers.edu}, %
 Krishna Anantha Padmanabhan\emailfoot{krishna.ananth@rutgers.edu}, %
 Jake Taubner\emailfoot{jdt97@scarletmail.rutgers.edu}, %
 Arlind Hoxha\emailfoot{ah621@scarletmail.rutgers.edu}%
 \\Teaching Assistant: Rahul Shome\emailfoot{rahulshome.in@gmail.com}%
 \\Department of Computer Science%
 \\Rutgers University}
\pdfauthor{author1}

\keywords{STAR, S2, Graphics, Introduction, Taxonomy, Body, Machine Learning, Planning, Graph Algorithms, Optimization}

\usepackage{amssymb}
\usepackage{amsmath}
\usepackage{pifont}
\usepackage[usenames, dvipsnames]{color}

\newcommand{\cmark}{\color{Green}{\ding{51}}}
\newcommand{\xmark}{\color{Red}{\ding{55}}}

\begin{document}

\maketitle

%\begin{abstract}
%  \paragraph{}
%\end{abstract}

%\begin{CRcatlist}
%  \CRcat{I.3.3}{Computer Graphics}{STAR Topic}{CRcat index}
%  \CRcat{I.3.7}{Computer Graphics}{STAR Topic}{CRcat index};
%\end{CRcatlist}

\keywordlist
\setlength{\parskip}{0pt}
\setlength{\parindent}{10pt}

\section{Introduction}

Planning in real time is a challenging task because of the dynamic
nature of environment. A major problem faced by researchers is in
finding an optimal strategy to decide when to expand the state space
and when to stop and recalculate. Another problem faced by researchers
is in expanding algorithms that work with local search to work for
global search where the number of variables increases by a huge
number.

The search domains in real world scenarios are non-deterministic,
complex and huge. This can be owed to the different types of agent’s
viz., characters, objects and obstacles and their behaviors that can
be present in a dynamic environment. Simulating such a domain
virtually is a challenging task due to the large branching factor and
high computational overhead that encompasses it. One of the principal
ways in which we can reduce this overhead is by making use of previous
experiences. This is where machine learning aids, since it
automatically learns, optimizes and plans the agent’s next move based
on previous moves. We need our algorithm to learn the properties of
the search domain and plan accordingly by making use of efficient
heuristic functions and cost functions.

Consider a humanoid robot that can move through any unknown
environment by avoiding all kinds of obstacles, much like C-3PO in
Star Wars. Such a robot would become the state of the art in robotics
and have enormous applications in all aspects of life. These robots
could save lives in the sectors of health, industry, military, etc.
The motion and trajectory planning of this robot would be the most
fundamental, yet formidable challenges of autonomous robotic design.
Since we would have high-dimensional state spaces, using a machine
learning algorithm for its planning stage would be very resource
friendly. The design of such an algorithm is very tricky since we
would have to consider all variables in the robot’s physics but if it
is achieved, we would have a robot that could change our lives
forever.

Machine learning in planning and scheduling can be applied anywhere an
expensive, domain-specific problem exists, and an inexpensive solution
is required; meaning, for businesses that want to make more money, and
academics that need to solve problems with less money. This money is
saved both by reducing required time and optimizing resource use (to
the point it is cost-effective to do so).

Commercial
\begin{itemize}
\item Near-optimal schedules
\item Factory worker scheduling
\item CNC Machine Operation (Lathes, Mills)
\item Production Lines
\item Dispatching
\item Job-shops
\item Analytics
\item Advertising
\item Code repair
\item Data mining
\end{itemize}
Academic
\begin{itemize}
\item Human Learning
\item Mathematical Programming
\item Integer programming problems
\item Empirical Analysis
\item Induction
\item Knowledge-based assistants
\item Geo-coding
\item Data search
\end{itemize}

There are various methods for machine learning and planning because of
its hefty and numerous requirements and features. Each method has its
own purpose and varies depending on the problem to be solved. In order
for machine learning and planning to take place, there needs to exist
a search space, initial and end goal states, and solution. A method
can be evaluated by the following criteria: planning time per move,
likeliness of finding optimal path or near-optimal path, learning
time, and memory usage. Ideally, we want to minimize planning time,
learning time, and memory usage while obtaining as close to optimal
solution as possible. With each of these criteria to keep track of
come with challenges on how to adjust and create a more powerful and
efficient algorithm for machine learning and planning.

The challenges of machine learning and planning is not that of how to
get a solution, but how fast one can get a solution, how efficient the
method is, and how optimal the solution is. There is an abundance of
methods for machine learning and planning, each crafted to solve a
different type of problem. Some methods such as the RTAA* follows
trajectories of smaller cost for given time limits per search because
it updates the heuristics quicker, which allows it to use larger local
search spaces, which overcompensates for its less informed heuristics.
While on the other hand, the min-max LRTA* is expected to run in a
safely explorable state space, meaning the state spaces have to be
relatively small or contain many evenly spread out goal states, but is
inefficient if otherwise. Real-time search does not focus on the
learning aspect but focuses on the time, thus outputting a sub-optimal
solution, but nonetheless a fast solution. Pattern Databases and
Macros can be used to improve performance by making it more efficient
to find states. Hierarchical Task Networks can also be used to help
organize planning algorithms for better efficiency. As one can
observe, there are various trade-offs in efficiency, performance, and
costs, depending on the algorithm's purpose.

As observed in the previous paragraph there are various methods that
are optimal for one scenario but not for another. Since there are a
plethora of problems and solutions under the umbrella of machine
learning and planning, there is no one "Holy Grail", but a "Holy
Grail" for each problem and solution. We can, however, speculate on
what a "Holy Grail" solution would entail. One must observe what is
important in a solution: does the solution need to be obtained as fast
as possible, or does the solution need to be optimal, etc.?
Observation of initial and end goal states along with search space is
also key. If the solution must be optimal and the search space is
small, using the RTAA* would be appropriate since it will output an
optimal solution, but the min-max LRTA* would be preferred over the
RTAA* since the LRTA* is the more optimal in smaller search spaces. A
Holy Grail solution would seek to combine the optimality of LRTA* with
the faster speed of RTAA*. Hierarchical Task Networks and Macro
Abstractions are also very useful in making planning algorithms faster
and more efficient. The downside to them is that creating a planning
model from a real-world problem is a difficult and time consuming task
for humans. A long term goal of planning is to create a model that
automatically creates and updates the formulation of a problem.
Therefore, the Holy Grail solution would also include the ability to
automatically and dynamically generate new abstraction models for the
planning, to help with non-deterministic environments.

\section{Prior Work}
The study of machine learning and planning can be broken down into
several categories: Pattern Databases, local search space, Real Time
Adaptive A*, and Learning Real Time A*. . Pattern database are
admissible heuristics that are studied and tested on
domain-independent search problems. Another criteria is the use of the
local search space to optimize large scale search problems, while the
use of different planning heuristics is tested and analyzed on
different domains.\cite{haslum2007domain} The primary algorithms of
focus are the Real Time Adaptive A* and Learning Real time A*.
Currently the industry has not fully utilized Machine Learning and AI
for navigation applications. Industry is currently using Incremental
search methods such as dynamic A* for unknown maps. This is used in
DARPA’s Unmanned Ground Vehicle program, Mars Rover, and games such as
Age of Empires 2.

PRODIGY is an architecture used in planning and
learning.\cite{stone1994need} The current frontier of Academic
research is currently focusing on many sections of machine learning
and planning. A study is being done on finding a heuristic method
which can best use machine learning to efficiently handle
domain-independent search problems. Some research is done on reducing
the complexity and the computation need of the algorithm. Another
research method is focused on optimizing this algorithm through aiding
parameters. Research is also being done to help the algorithm solve
larger problems more successfully and expanding its optimal solutions
from the local space into a larger space. Current problems with the
application of machine learning in industry are the difficulty in
extending this problem to multiple monotonous agents. Another problem
is on the balance between exploration and exploitation where most
algorithms have an excessive computation complexity for the optimal
path when good is enough.

\section{Future Work}
Since there exists no "perfect" planning algorithm, many open problems
and opportunities for future research exist. One area is pattern
databases; the downfall of pattern databases is that they require a
great deal of memory. Future work to compress pattern databases and
implement them in more efficient and less memory consuming data
structures would be highly useful. RTAA* (Real Time Adaptive A*) could
be extended to work with inconsistent heuristics as well as be
combined with other learning methods such as HTNs. Another area for
future work is using the Macro-FF learning method with complex
structures such as HTNs. Not much research has been done with learning
and HTNs, making this an important area for future research. A long
term open problem is automatic abstraction of planning domains and
problems, as explained in the "Holy Grail" solution. Choosing a good
abstraction level for a planning model is a difficult process for a
human, so methods to automatically develop and update the formulation
of a problem is an important area for future work.

\begin{table*}
  \begin{center}
  \begin{tabular}{|c|c| c |c|c|c|}
    \hline
     & Fast at runtime & Memory Efficient & Real-time & Scalable & Non-deterministic domains\\
     \hline
    RTAA* & \cmark & \cmark & \cmark & \xmark  & \cmark \\
    \hline
    LRTA* & \cmark & \cmark & \cmark & \xmark  & \xmark \\
    \hline
    Pattern Databases & \cmark & \xmark & \xmark & \cmark & \xmark\\
    \hline
    HTN   & \xmark & \cmark & \xmark & \cmark & \xmark \\
    \hline
    Macro Actions & \cmark & \cmark & \xmark & \cmark & \xmark \\
    \hline
    Strong AI & \cmark & \cmark & \cmark & \cmark & \cmark\\
    \hline
  \end{tabular}
  \end{center}
  \caption{Feature matrix.}
  \label{table:feature_matrix}
\end{table*}

\section{Overview of Data Structures and Representations}
The entire task of planning in an unknown domain can be broken down
into the following sequence of steps. The first and foremost challenge
is to choose an efficient and space conservative method to store the
data. Since the domains can be large and unknown, the size of the
problem can grow exponentially. This problem escalates even more if
the environment is dynamic. The next challenge is to reach the goal
state from a given state in finite time with the least possible cost.
These solutions will be deployed in real time and it is of utmost
importance that they carry out the action in the given timeframe. If
the environment is not fully observable, then the agents also have to
take into account the probability of change in the portion of the
environment that is not observable. To make planning feasible, we will
take into consideration the data structures and memory
representations.

\subsection{Hierarchical Task Networks}

A Hierarchical Task Network is used to represent dependencies among
actions. The world is represented by a set of states and actions. The
actions represent a transition from one state to another. The reason
why a hierarchical task network is preferred is because it simplifies
the tasks to be performed by decomposing complex tasks into a simple
sequence of primitive actions. To notify the agent on how to decompose
non primitive tasks into subtasks, a set of methods are utilized for
decomposing a particular kind of task into a set of subtasks provided
a set of preconditions are satisfied. For each task, there may be more
than one applicable method, and thus more than one way to decompose
the task into subtasks. For example, in a grid
world representation of a room that has to be cleaned by a vacuum
cleaner agent, a hierarchical task network for the action of cleaning
a room can be decomposed into a set of actions of suck, move right and
move left where the move actions become valid when the grid cell is
clean. One of the task planners that utilize HTNs is the Simple
Hierarchical Ordered Planner (SHOP). A SHOP is domain independent. HTN
planning is in essence a case of decomposition of the entire problem
into subtasks, stopping when it reaches primitive tasks. An ordered
task planner plans in the same way the sequence of actions will be
executed. This reduces the complexity and assumes the deterministic
nature of the environment

\subsection{Pattern Databases}

With the complex actions having been decomposed to a basic set of
primitive tasks, one important data structure that reduces the
complexity exponentially would be pattern databases. A pattern
database holds a collection of solutions to sub problems which have to
be solved to get the overall solution. If the amount of space
available is higher, then the solutions to the sub problems can be
precomputed and stored for more efficient and faster calculations.
Essentially, pattern databases work by the comparison of current state
of the decomposed problem to the problem indexed in the database and
returns the precomputed solution.

\subsection{Macro Actions}

To mutate the current state of the environment or agent to the closest
possible recognizable state in the pattern database, a sequence of
actions have to be performed. A macro-action is a meta-action which is
computed from a sequence of action steps. In a planning environment
that involves progressive forward chaining, application of a macro
action to a state produces a successive state that can be obtained by
performing a sequence of actions. In a path finding world, this can be
essentially thought of as an act of extending the path towards the
solution. A good choice leads to an increase in agent performance
while a poor choice raises the branching factor thereby causing the
performance to drop. One of the planners that utilize macros is the
Macro FF planner. An FF planner is a fully automated planner that uses
a heuristic search approach to estimate the best node/state to expand
to reach its goal. The utilization of macros reduces the expansion
factor further since it learns which states to expand and which not
to.

\section{Overview of Search Techniques}


To find out the node that has to be expanded next in the path towards
the goal, the choice has to be made such that the expanded nodes take
us closer to the goal and also compute that optimal solution quickly.
A* search algorithm does give an optimal solution but one major
drawback is the inability to deploy that in very large state spaces
since it would run out of memory before the goal can be reached. Some
variations of A* such as Real Time Adaptive A* and the Learning in
Real Time A* algorithms can be considered viable alternatives
depending on the environments in which the planner has to work on.

\subsection{Real Time Adaptive A*}

The basic idea behind Real Time Adaptive A* is simple. In an
environment with multiple agents, each agent has to perform A*
repeatedly with the same goal state but with possibly different start
states. But each time A* is performed, the heuristics give a better
estimate so that future A* searches are quicker. A* in general works
by using \(f(s)\) which is the sum of the cost function \(g(s)\) from
start node to current node and heuristic estimate \(h(s)\) from the
current node to the goal state. A priority queue maintains a list of
open nodes and from these set of nodes, the one with the least
\(f(s)\) is chosen for expansion. To make future A* searches faster,
let us consider how the heuristic can be updated. Let \(s\) be a state
expanded during the search process. Obtaining an admissible estimate
for \(gd[s]\) is easy. The distance from the start state
\(s_\text{current}\) to any goal state via state \(s\) is equal to the
sum of distance from the start state \(s_\text{current}\) to state
\(s\) and the goal distance \(gd[s]\) of state \(s\). This is larger
than the goal distance of \(s_\text{current}\). If \(s'\) is the goal
state, then goal distance \(gd[s]\) of state \(s\) is larger than the
goal distance \(gd[scurrent]\) minus the distance from the start state
\(s_\text{current}\) to state \(s\).

\[ g[s] + gd[s] \geq gd[scurr] \]
\[ gd[s] \geq gd[scurr] - g[s] \]
\[ gd[s] \geq f[s'] - g[s] \]

Eventually, the heuristic proposed in this technique is
\[ h[s] = f[s'] - g[s] = g[s'] + h[s'] - g[s] \]

RTAA* doesn’t work when heuristics are inconsistent.

\subsection{Learning Real Time A*}

LRTA* updates the heuristics in a very basic way. For a given state
\(s\), LRTA* with a lookahead of one considers immediate neighbors.
For each neighbor state, the values of \(f\) is calculated. If \(s'\)
is the state with the least \(f\) value, and if it is greater than the
h value of the current state \(s\), then \(h(s)\) is initialized to
\(f(s')\). As far as LTRA* is concerned, the factors that have to be
taken into consideration are deeper lookaheads, heuristic weights and
adding backtracking so that the heuristics of predecessor nodes can
also be updated. For non-deterministic domains, min-max LRTA* is
proposed

\bibliographystyle{acmsiggraph}
\nocite{*}
\bibliography{S2bibliography.bib}

\end{document}
